%%%%%%%%%%%%%%%%%%%%%%%%%%%%%%%%%%%%%%%%%
% Structured General Purpose Assignment
% LaTeX Template
%
% This template has been downloaded from:
% http://www.latextemplates.com
%
% Original author:
% Ted Pavlic (http://www.tedpavlic.com)
%
% Note:
% The \lipsum[#] commands throughout this template generate dummy text
% to fill the template out. These commands should all be removed when 
% writing assignment content.
%
%%%%%%%%%%%%%%%%%%%%%%%%%%%%%%%%%%%%%%%%%

%----------------------------------------------------------------------------------------
%	PACKAGES AND OTHER DOCUMENT CONFIGURATIONS
%----------------------------------------------------------------------------------------

\documentclass{article}

\usepackage{fancyhdr} % Required for custom headers
\usepackage{lastpage} % Required to determine the last page for the footer
\usepackage{extramarks} % Required for headers and footers
\usepackage{graphicx} % Required to insert images
\usepackage{lipsum} % Used for inserting dummy 'Lorem ipsum' text into the template
\usepackage{mathtools}

% Margins
\topmargin=-0.45in
\evensidemargin=0in
\oddsidemargin=0in
\textwidth=6.5in
\textheight=9.0in
\headsep=0.25in 

\linespread{1.1} % Line spacing

% Set up the header and footer
\pagestyle{fancy}
\lhead{\hmwkAuthorName} % Top left header
\chead{Discussion enrolled: \hmwkClass\\Discussion Attended: \hmwkClass} % Top center header
\rhead{Page\ \thepage\ of\ \pageref{LastPage}} % Bottom right footer
\cfoot{\hmwkTitle} % Bottom center footer
\renewcommand\headrulewidth{0.4pt} % Size of the header rule
\renewcommand\footrulewidth{0.4pt} % Size of the footer rule


%----------------------------------------------------------------------------------------
%	DOCUMENT STRUCTURE COMMANDS
%	Skip this unless you know what you're doing
%----------------------------------------------------------------------------------------

% Header and footer for when a page split occurs within a problem environment
\newcommand{\enterProblemHeader}[1]{
\nobreak\extramarks{#1}{#1 continued on next page\ldots}\nobreak
\nobreak\extramarks{#1 (continued)}{#1 continued on next page\ldots}\nobreak
}

% Header and footer for when a page split occurs between problem environments
\newcommand{\exitProblemHeader}[1]{
\nobreak\extramarks{#1 (continued)}{#1 continued on next page\ldots}\nobreak
\nobreak\extramarks{#1}{}\nobreak
}

\setcounter{secnumdepth}{0} % Removes default section numbers
\newcounter{homeworkProblemCounter} % Creates a counter to keep track of the number of problems

\newcommand{\homeworkProblemName}{}
\newenvironment{homeworkProblem}[1][Problem \arabic{homeworkProblemCounter}]{ % Makes a new environment called homeworkProblem which takes 1 argument (custom name) but the default is "Problem #"
\stepcounter{homeworkProblemCounter} % Increase counter for number of problems
\renewcommand{\homeworkProblemName}{#1} % Assign \homeworkProblemName the name of the problem
\section{\homeworkProblemName} % Make a section in the document with the custom problem count
\enterProblemHeader{\homeworkProblemName} % Header and footer within the environment
}{
\exitProblemHeader{\homeworkProblemName} % Header and footer after the environment
}

\newcommand{\problemAnswer}[1]{ % Defines the problem answer command with the content as the only argument
\noindent\framebox[\columnwidth][c]{\begin{minipage}{0.98\columnwidth}#1\end{minipage}} % Makes the box around the problem answer and puts the content inside
}

\newcommand{\homeworkSectionName}{}
\newenvironment{homeworkSection}[1]{ % New environment for sections within homework problems, takes 1 argument - the name of the section
\renewcommand{\homeworkSectionName}{#1} % Assign \homeworkSectionName to the name of the section from the environment argument
\subsection{\homeworkSectionName} % Make a subsection with the custom name of the subsection
\enterProblemHeader{\homeworkProblemName\ [\homeworkSectionName]} % Header and footer within the environment
}{
\enterProblemHeader{\homeworkProblemName} % Header and footer after the environment
}
   
%----------------------------------------------------------------------------------------
%	NAME AND CLASS SECTION
%----------------------------------------------------------------------------------------

\newcommand{\hmwkTitle}{Assignment 1} % Assignment title
\newcommand{\hmwkClass}{Section 1D} % Course/class
\newcommand{\hmwkAuthorName}{Prianna \underline{Ahsan}} % Your name


\begin{document}

%----------------------------------------------------------------------------------------
%	PROBLEM 1
%----------------------------------------------------------------------------------------

% To have just one problem per page, simply put a \clearpage after each problem

\begin{homeworkProblem}
True or false? In every instance of the Stable Matching Problem, there is a stable matching containing a pair (m,w) s.t. m is ranked first on w's preference list and w is ranked first on m's preference list.
\vspace{10pt} % Question


This is clearly false, as evidenced on page 5 of [KT]. In this example, we have the following preference lists:\\
\setlength\parindent{25pt}
\indent m: (w, w')\\
\indent m': (w', w)\\
\indent w: (m', m)\\
\indent w': (m, m')\\
There are two sets of stable matchings, $S_{1}$ = {(m, w), (m', w')} and $S_{2}$ = {(m', w), (m, w')}. Neither of them contain a pair s.t. both m and w are first on each other's preference lists.

\end{homeworkProblem}

%----------------------------------------------------------------------------------------
%	PROBLEM 2
%----------------------------------------------------------------------------------------

\begin{homeworkProblem} % Custom section title
True or false? Consider an instance of the Stable Matching Problem in which there exists a man, m, and a woman, w, s.t. m is ranked first on w's preference list and w is ranked first on m's preference list. Then in every stable matching S for this instance, the pair belongs to S.% Question
\vspace{10pt}
%--------------------------------------------

Yes, by definition of "stable matching", this must be true. A situation in which this is not true creates an instability.


\end{homeworkProblem}

%----------------------------------------------------------------------------------------
%	PROBLEM 3
%----------------------------------------------------------------------------------------

\begin{homeworkProblem}% Roman numerals
Such an algorithm doesn't exist. This can be proved by a counter example in which the two networks have the following ratings for shows (1 is the best, 6 is the worst):\\
\setlength\parindent{25pt}
\indent A: (1, 3, 5)\\
\indent B: (2, 4, 6)\\
Say A releases a schedule S = ( 1, 3, 5 ) and B releases a schedule T = ( 2, 6, 4 ). Then network A wins 2 slots and network B wins 1 slot. This pair of schedules is not stable, because B will want to rearrange T to T' = ( 6, 2, 4) so that they can win two slots. Similarly, if T = ( 6, 2, 4 ), network A will want to rearrange S to S' = ( 5, 1, 3 ) so that they can win all of the slots.

\end{homeworkProblem}

%----------------------------------------------------------------------------------------

%----------------------------------------------------------------------------------------
%	PROBLEM 4
%----------------------------------------------------------------------------------------

\begin{homeworkProblem}
\begin{homeworkSection}{Description} % Section within problem
Since each ship visits ports in some order, and each port is visited by each ship in some order, we can set up sets P and S of n ports and n ships, respectively, where each $s_{i} \in S$ has an n-tuple of ports in ascending chronological order, and each $p_{i} \in P$ has an n-tuple of ships in descending chronological order (i.e., the first port in $s_{i}'s$ list is the first port visited by $s_{i}$ but the first ship in $p_{i}'s$ list is the last/most recent ship to visit $p_{i}$). Now, we can use the algorithm described in figure 1.2 on page 6 of [KT] to determine which ship stops at which port for maintenance.


\end{homeworkSection}
%--------------------------------------------

\begin{homeworkSection}{Solution} 
\problemAnswer{ % Answer
While there is a ship s which is not assigned to remain at a port,\\
\setlength\parindent{25pt}
\indent Choose such a ship s.\\
\indent Let p be the highest ranked port in s's preference list that has not yet\\
\indent been assigned a ship.\\
\indent If p is free,\\
\indent \indent assign s to p.\\
\indent Else, p is currently assigned to s'.\\
\indent \indent If p prefers s' to s then,\\
\indent \indent \indent s remains unassigned.\\
\indent \indent Else,\\
\indent \indent \indent assign s to p,\\
\indent \indent \indent s' becomes unassigned.\\
Return M, the set of matched ships and ports.
}
\end{homeworkSection}

%--------------------------------------------

\begin{homeworkSection}{Analysis}
Since this solution is simply Gale-Shapely's stable matching algorithm (figure 1.2, page 6 in [KT]), and we know that no two ships can stop at the same port at the same day (i.e., each ship and port's preference list is unique), the set M consists of n unique order pairs (s, p) for  all $s \in S$ and $p \in P$. Like the Gale-Shapely algorithm, the main while-loop in the above solution takes $n^{2}$ steps to terminate.
\end{homeworkSection}

%--------------------------------------------

\end{homeworkProblem}


%----------------------------------------------------------------------------------------
%	PROBLEM 5
%----------------------------------------------------------------------------------------

\begin{homeworkProblem} % Roman numerals

\begin{homeworkSection}{Description} % Section within problem
The problem asks us to maximize calories, subject to some weight constraint. Essentially, our function maximizes a sub-function that returns a linear combination of number of fluid packs and calories per pack, subject to the constraint that the total weight of the packs does not exceed some weight value, $w_{max}$. Let $\vec{c} := <c_{1},c_{2},\cdots, c_{n}>$ represent the vector of calories per pack, and $\vec{w} := <w_{1},w_{2},\cdots, w_{n}>$ represent the vector of weights per pack. Then, each sub-solution $S_{i}$ is assigned some vector $\vec{q} := <q_{1},q_{2},\cdots, q_{n}>$ representing the quantities of each pack used in that sub-solution. Since there are $ w_{max}+n \choose n$ possible solutions to the equation $w_{1}q_{1}+ \cdots +w_{n}q_{n} \leq w_{max}$, it makes sense to recursively break our problem down into sub-problems that each return optimal sub-solutions. Then, our maximization function can examine each sub-solution to find the maximal possible caloric value obtained.
\end{homeworkSection}


\begin{homeworkSection}{Solution} % Section within problem	
\problemAnswer{ % Answer
// The solution takes a vector of quantities, $\vec{q}$, and an integer\\
// weight, as input. It returns a vector of integer quantities, $\vec{q}$. The\\
// vectors of calories and weights are assumed to be defined globally.\\
// Initially, the solution is called with weight set to $w_{max}$ and $q_{i} = 0\ \forall\ q \in \vec{q}$.\\
// This solution is provided without proof.\\
\setlength\parindent{25pt}
If weight is 0,\\
	\indent return $\vec{q}$;\\
For i = 0 to n:\\
	\indent If $w_{i} \leq weight$,\\
	\indent \indent	$q_{i} = q_{i}+1$,\\
	\indent \indent	$\vec{S}_{i}$ = Solution($\vec{q}$, weight-$w_{i}$).\\
	\indent Else,\\
		\indent \indent $S_{i} = \vec{q} $.\\
Let $\vec{S}_{max} = \vec{q}$, let $cal_{max} = 0$.\\
For each $\vec{S}_{i} \in \vec{S}$:\\
	\indent If the first element in $\vec{S}_{i}$ is 0,\\
	\indent \indent Leave $\vec{S}_{max}$ unchanged.\\
	\indent Else,\\
	\indent \indent $cal_{i}$ = sum of $\vec{S}_{i} \times \vec{c}$.\\
	\indent \indent If $cal_{i} > cal_{max}$,\\
	\indent \indent \indent $cal_{max} = cal_{i}$,\\
	\indent \indent \indent $\vec{S}_{max} = \vec{S}_{i}$.\\
return $\vec{S_{max}}$.
}
\end{homeworkSection}

\begin{homeworkSection}{Analysis} % Section within problem
The first for-loop creates a sub-problem thread consisting of solutions beginning with the $i^{th}$ fluid pack. If the $i^{th}$ fluid pack exceeds $w_{max}$, the loop moves on to the next pack. If any of the individual fluid pack weights exceed $w_{max}$, no solution sequence will exist that begins with that fluid pack (i.e., no good solution sequences exist that begin with 0). In the worst case, n solution sequences are generated by this loop. The second for-loop ignores all solutions that begin with 0 (empty solutions) and compares the maximum calories attained in the $i^{th}$ solution with the current maximum calorie value. This comparison operation takes n steps per step of the second loop, so this for-loop takes $n^{2}$ steps in the worst case. Thus, each call of the solution takes $n^{2}+n$ steps in the worst case.
\end{homeworkSection}

\begin{homeworkSection}{Proof of Correctness} % Section within problem
At present, I'm not positive that this solution is correct, thus I certainly cannot prove its correctness. Given more time, I could probably come up with something resembling an inductive proof, but as of this moment, no proof exists.
\end{homeworkSection}
\end{homeworkProblem}

%----------------------------------------------------------------------------------------

\end{document}
